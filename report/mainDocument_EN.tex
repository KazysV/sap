% Kompiuterijos katedros ir kibernetinio saugumo laboratorijos šablonas
% Template of Department of Computer Science II or cybersecurity laboratory
% Versija 1.4 2024 m. sausi [ March, 2015]
% comments/bug fixes send to template authors: Linas Bukauskas or Agnė Brilingaitė
\documentclass[a4paper,12pt,fleqn]{article}
\input{allPacks}
\usepackage{graphicx}
\newtoggle{inLithuanian}
 %If the report is in Lithuanian, it is set to true; otherwise, change to false
\settoggle{inLithuanian}{false}

%create file preface.tex for the preface text
%if preface is needed set to true
\newtoggle{needPreface}
\settoggle{needPreface}{false}

\newtoggle{signaturesOnTitlePage}
\settoggle{signaturesOnTitlePage}{false}

\input{macros}

\begin{document}
 % #1 -report type, #2 - title, #3-7 students, #8 - supervisor
 \depttitlepage{Signal Analysis and Processing}{Moving average algorithm analysis\\{\small }}{Kazimieras Vitkus} 
 {}{}{}{}% students 2-5
{dr. Tadas Meskauskas}

\tableofcontents
\newpage


\section{Introduction}

\hspace{1em} The task of this project is to implement, analyse and compare moving average algorithms. Two algorithms
are in the scope of the project: Weighted Moving average (WMA) and Exponential Moving Average (EMA). The
algorithms will be implemented in Python programming language, using NumPy, Pandas, Matplotlib libraries. The
programming will be done in Jupyter Notebook environment. The algorithms will be tested on the same dataset.


\section{Definition of algorithms and signal}

\subsection{Moving Average algorithm}\label{MA}
\hspace{1em} A moving average algorithm is a statistical technique used to analyze sequential data points by creating 
a series of averages of different subsets of the full dataset. Typically applied in time-series analysis, 
finance, and signal processing, it smooths out fluctuations to reveal underlying trends or patterns. 
The algorithm involves calculating the mean of a specified number of data points within a sliding window,
which moves along the dataset, updating the average at each step. This process effectively filters out 
short-term fluctuations, making it easier to identify longer-term trends or patterns within the data.

Given that we have a digital signal \begin{math}
    f_0,f_1,...,f_N
\end{math} and chosen the window size \begin{math}
    K
    \end{math} - positive integer. By applying the moving average algorithm to \begin{math}
        f_i
    \end{math} we get the sum of averages in window of size \begin{math}
        K
    \end{math} to both side. The formula for moving average is:
    \begin{gather}\label{MA_equation}
        g_i = \frac{1}{2K+1} \sum_{j=-K}^{K} f_{i+j},\hspace{1 cm}i = K, K+1, ..., N-K
    \end{gather}
   
    To be noted, the first value of signal that has been averaged value is \begin{math}
        f_K
    \end{math} and the last value is \begin{math}
        f_{N-K}
        \end{math}. The moving average algorithm is used to smooth out short-term fluctuations in the data, 
        however during the process the length of the signal is reduced.
        \subsection{Weighted Moving Average algorithm}\label{WMA}
        \hspace{1em} In the subsection \ref{MA} moving average algorithm was introduced. The moving average 
        algorithm is simple, due to the fact that it averages all the values in the window equally as shown 
        in the formula \ref{MA_equation}. However, in some cases it is more beneficial to give more weight to
        the values that are closer to the current value. This is where the weighted moving average algorithm
        comes in. The weighted moving average algorithm is a variation of the moving average algorithm, where
        the values in the window are given different weights. The formula for weighted moving average is:
        \begin{gather}\label{WMA_equation}
            g_i = \frac{1}{2K+1} \sum_{j=-K}^{K} w_j f_{i+j},\hspace{1 cm}i = K, K+1, ..., N-K, \hspace{1 cm} w_j \sum_{j=-K}^K = 1
        \end{gather}


        The weighted moving average algorithm weights can be defined in various ways. The most common way is to
        use the Gaussian function. The Gaussian function is a bell-shaped curve that is symmetric around the mean.
        The Gaussian function is defined as:
        \newpage
        \begin{gather}\label{Gaussian}
            p(x) = \frac{1}{\sqrt{2\pi}}e^{-\frac{x^2}{2}}
        \end{gather}

        Lets say that \begin{math}
            f_{i-K}
        \end{math} and \begin{math}
            f_{i+K}
        \end{math} 
        should be \begin{math}
            L
        \end{math} times less important than the center value \begin{math}
            f_i
        \end{math}. In this case \begin{math}
            L > 1
        \end{math} is a positive integer, a parameter to weighted moving
        average algorithm. The range of weights distribution could be 
        discretised in range: \begin{math}
            -x_k \leq x \leq x_k
        \end{math}, then it can be calculated where \begin{math}
            x_k \end{math}, is less than center value \begin{math}
                x_0 \end{math} in Gaussian distribution formula \ref{Gaussian}:
                \begin{gather}\label{Gaussian_discretised}
                    \frac{p(0)}{p(x_k)} = L, 
                    \hspace{1 cm} x_k = \sqrt{-2\ln{L}}
                \end{gather}


        To comply with the rule that weights \begin{math}
            \sum_{j=-K}^K w_j = 1
        \end{math} the weights should be normalized. The normalized weights are defined as:
        \begin{gather}\label{Normalized_weights}
            w_j = \frac{p(j)}{\sum_{j=-K}^K p(j)}, \hspace{1 cm} j = -K, -K+1, ..., K
        \end{gather}
        \begin{figure}[ht]
            \centering
            \includegraphics[width=0.8\textwidth]{images/weights.png} % Replace "path/to/your/image" with the actual path to your image file
            \caption{Distribution of normalized weights where L = 10 and K = 30}
            \label{fig:my_image}
        \end{figure}

        \subsection{Exponential Moving Average algorithm}\label{EMA}
        \hspace{1em} The exponential moving average algorithm is another variation of the moving average algorithm.
        This method is similar to the weighted moving average algorithm \ref{WMA}, 
        but have two main diffrences:
        \begin{enumerate}
            \item Instead of parameters \begin{math}
                K
            \end{math} and \begin{math}
                L
            \end{math} the exponential moving average algorithm has only one parameter \begin{math}
                \alpha
            \end{math}.
            \item When new signal value is appended, it is enough to know 
            the previous average value to calculate the new average value. For refrence, 
            other averaging algorithms need to know all the values in the window.
        \end{enumerate}

        To average the signal with exponential moving average algorithm, 
        the formula is used: \begin{gather}
            g_n = (1-\alpha) g_{n-1} + \alpha f_n, \hspace{1 cm} n = 1, 2, \dots , \hspace{1 cm} g_0 = f_0
        \end{gather}
        
        \subsection{Signal for the algorithm analysis}
        \hspace{1em}For the task of this project, the financial market signal was chosen. The daily values of 
        the S\&P 500 index\cite{SP500} at market close from 2014 to 2024 were used. The signal was chosen because it has
         high volatility and represents the financial market well. 


        \begin{figure}[ht]
            \centering
            \includegraphics[width=0.8\textwidth]{images/SP500.png} % Replace "path/to/your/image" with the actual path to your image file
            \caption{S\&P 500 index signal from 2014 to 2024}
            \label{fig:SP500}
        \end{figure}

        As can be seen from figure \ref{fig:SP500}, the value of the index have increased over 2.5 times in the
        period of 10 years. The signal has high volatility, which is good for the analysis of moving average algorithms.
        
        
        
        \newpage
        \section{Implementation}
\hspace{1 em} All of the algorithms were implemented in Python programming language, using libraries
such as NumPy, Pandas, Matplotlib. The programming was done in Jupyter Notebook environment. 

Appendix 1 contains the full Jupyter Notebook file with the implementation and analysis of the algorithms. 

        \subsection{WMA. Weights}
        \begin{verbatim}
def gaussian_distribution(x):
    return 1/(np.sqrt(2*np.pi)) * np.exp(-((x**2)/2))      
    
    
def get_weights(L,K):

#boundary value for weights
    x_k = np.sqrt(2*np.log(L))

#creating X axis for gaussian weight distribution
    K_j = np.linspace(-K,K,2*K+1)


#distributing weights along X axis
    x_j = [x_k * j/K for j in K_j]

# applying gaussian distribution to weights
    p_j = [gaussian_distribution(x) for x in x_j]

#normalization of weights
    w_j = [p/sum(p_j) for p in p_j]
    return K_j,w_j,p_j
        \end{verbatim}
\newpage
        \subsection{WMA. Algorithm}
        \begin{verbatim}

def apply_weights(data, weights):
        weighted_data = 0
        for i in range(len(data)):
            weighted_data += data[i] * weights[i]
        return weighted_data

def weighted_moving_average(data, L, K):
    
    K_j,weights,p_j= get_weights(L,K)
    weighted_data = np.zeros(len(data))
    
    for i in range(K, len(data) - K):
        window_data = data[i - K : i + K + 1]
        weighted_data[i] = apply_weights(window_data, weights)
    
    weighted_data = np.where(weighted_data == 0, np.nan, weighted_data)
    
    return weighted_data
            \end{verbatim}
        \subsection{EMA. Algorithm}
        \begin{verbatim}
def exponential_moving_average(data, alpha):
    averaged_data = np.zeros(len(data))
    averaged_data[0] = data[0]
    
    for i in range(1, len(data)):
        averaged_data[i] = alpha * data[i] + (1 - alpha) * averaged_data[i - 1]
        
    return averaged_data
            \end{verbatim}

        \newpage
        \section{Analysis}

        \subsection{Averaging signal with WMA}

        \hspace{1em} The weighted moving average algorithm has two parameters that can be adjusted

        \begin{enumerate}
                \item \begin{math} \bold{K}\end{math} - the size of the window;
                \item \begin{math}
                    \bold{L}
                \end{math} - the parameter that defines the importance of the values in the window.

                \end{enumerate}

                By adjusting these parameters, the algorithm tends to smooth out the signal more or less.
                However, by over tuning the parameters, the algorithm can lose the important information in the signal, i.e.
                the algorithm can over smooth the signal.
                \begin{figure}[ht]
                    \centering
                    \includegraphics[width=1\textwidth]{images/WMA_ex_1.png} % Replace "path/to/your/image" with the actual path to your image file
                    \caption[Short caption for the list of figures]{On top left chart original signal is shown, on top right chart signal is 
                    averaged with WMA with K = 30 and L = 10, on bottom left chart signal is averaged with WMA  with K = 270 and L = 30, on bottom right chart 
                    signal is averaged with WMA  with K = 5 and L = 5.}
                    \label{fig:WMA_ex_1}
                \end{figure}

                As can be seen from the figure \ref{fig:WMA_ex_1}, the parameters adjust the smoothness of the signal. 
                The larger the \begin{math}
                    K
                \end{math} the smoother the signal gets, also the window of values get shorter,due to 
                algorithms design that only allows to compute values that are in range of
                \begin{math}
                    g = (f_{0+K}; f_{N-K})
                \end{math}
                , see the top left and 
                the bottom left charts in figure \ref*{fig:WMA_ex_1}.
                 The larger the \begin{math}
                    L
                \end{math} the more the values in the window are weighted towards the center value.

                For this particular case study of SP500 signal, the logic parameters for WMA should correlate with the cycles
                of weeks, months, quarters and years. For the case of WMA, parameter \begin{math}
                    K
                \end{math} represents the days that are in window, for example if we set \begin{math}
                    K = 30
                \end{math} the window will be 60 days, which is approximately one month before and one month after to
                the date of intrest.

                \subsection{Qualitative analysis of WMA}

                \hspace{1 em} The main concerns with moving average algorithms 
                is that they tend to lag behind the signal, can be too smooth or retain too much noise. It all depends on
                the parameters that are chosen for the algorithm. The weighted moving average algorithm is more flexible, 
                because it allows to adjust the weights of the values in the window.

                There some methods to evaluate the quality of the algorithm:
                \begin{enumerate}
                    \item Visual inspection of the signal;
                    \item Comparison of the averaged signal with the original signal;
                    \item Evaluating the correlation between the original signal and the averaged signal;
                    \item Evaluating the gradient variance and covariance of the original signal and the averaged signal.
                \end{enumerate}
                \begin{figure}[ht]
                    \centering
                    \includegraphics[width=1\textwidth]{images/WMA_ex_2.png} % Replace "path/to/your/image" with the actual path to your image file
                    \caption{In top chart the variance of original signal is shown with variance of = 566.397,
                    in the bottom chart the variance of signal averaged with WMA with K = 30 and L = 10 is shown with variance of = 19.677.}
                \label{fig:WMA_ex_2}
                \end{figure}

                As depicted in figure \ref{fig:WMA_ex_2}, the variance of the signal is reduced by the weighted moving average algorithm.
                The gradient of the signal variance, can help to evaluate if the smoothness of the signal fits the requirements.
                The gradient of the variance can be calculated as:
                \begin{gather}
                    \Delta f = \frac{f_{n+1} - f_n}{f_n}
                \end{gather}

                To evaluate if the averaged signal retains the trends, the correlation between the original signal and the averaged signal 
                can be calculated. The correlation can be calculated as:
                \begin{gather}
                    \rho = \frac{cov(f,g)}{\sqrt{var(f)var(g)}}
                \end{gather}

                \begin{table}[ht]
                    \centering
                    \begin{tabular}{|c|c|c|c|c|c|}
                        \hline
                        & \textbf{K=7} & \textbf{K=30} & \textbf{K=90} & \textbf{K=180} & \textbf{K=360} \\
                        \hline
                        \textbf{L=1} &0.951&0.814 &0.583 &0.391 &0.209 \\
                        \hline
                        \textbf{L=5} &0.951 &0.815 &0.583&0.395 &0.221\\
                        \hline
                        \textbf{L=10} &0.951 &0.815 &0.583&0.396 &0.224 \\
                        \hline
                        \textbf{L=20} &0.951 &0.815 &0.583&0.397&0.227 \\
                        \hline
                        \textbf{L=30} &0.951 &0.816 &0.583& 0.397&0.229 \\
                        \hline
                        \textbf{L=50} &0.951 &0.816 &0.584 &0.398 &0.231 \\
                        \hline
                    \end{tabular}
                    \caption{Matrix of WMA parameters K and L impact on correlation}
                    \label{tab:KL_values}
                \end{table}

                As could be seen from the table \ref{tab:KL_values}, the correlation between the original signal and the averaged signal
                is dependant on the parameter \begin{math}
                    K
                \end{math}. The parameter \begin{math}
                    L
                    \end{math} has negligible impact on the correlation.

                \begin{figure}[ht]
                    \centering
                    \includegraphics[width=1\textwidth]{images/WMA_ex_3.png} % Replace "path/to/your/image" with the actual path to your image file
                    \caption{S\&P 500 signal from 2014 to 2024 overlayed with signal averaged with WMA with K = 30 and L = 10}
                    \label{fig:WMA_ex_3}
                \end{figure}
\newpage
                \subsection{Averaging signal with EMA}

                \hspace{1 em} EMA has only one parameter \begin{math}
                    \alpha
                \end{math}, which is the weight of the new value in the signal. However, it retains the same
                issues as WMA - if the weight is not chosen correctly, the algorithm can lag behind the signal, be too smooth or retain too much noise.

                \begin{figure}[ht]
                    \centering
                    \includegraphics[width=1\textwidth]{images/EMA_ex_1.png} % Replace "path/to/your/image" with the actual path to your image file
                    \caption{On top left chart original signal is shown, on top right chart EMA is applied with $\alpha = 0.1$, on bottom left chart EMA is applied with $\alpha = 0.01$, on bottom right chart EMA is applied with $\alpha = 0.001$.}
                    \label{fig:EMA_ex_1}

                \end{figure}

                As can be seen from the figure \ref{fig:EMA_ex_1}, the smaller the \begin{math}
                    \alpha
                \end{math} the smoother the signal gets and lag occurs.

                Although the EMA algorithm has only one parameter and is more difficult to fine tune, but it has an 
                advantage over the WMA algorithm - it does not require all the values in the window to calculate the new average value. And 
                even next value can be predicted if previous value is known.
                \newpage
                \subsection{Qualitative analysis of EMA}

                The EMA algortithm tends to be more tough for fine tuning, but the same evaluation techniques can
                be adapted for EMA as for WMA. The correlation between the original signal and the averaged signal can be calculated as:
                
                \begin{table}[ht]
                    \centering
                    \begin{tabular}{|c|c|c|}
                        \hline
                        Alpha & Covariance & Correlation \\
                        \hline
                        0.1 & 51.67 & 0.997\\
                        0.08 & 40.53 & 0.996 \\
                        0.06& 29.38 & 0.995 \\
                        0.04& 18.43& 0.993 \\
                        0.02& 8.29 & 0.988 \\
                        0.01& 3.72 & 0.979 \\
                        0.005& 1.50 & 0.967 \\
                        0.001 & 0.22 & 0.944 \\
                       
                        \hline
                    \end{tabular}
                    \caption{Covariance and correlation between original signal and signal averaged with EMA with different $\alpha$ values}
                    \label{tab:correlation_covariance}
                \end{table}
                
            As can be seen from the table \ref{tab:correlation_covariance}, the EMA algorithm tends to retain 
            the trend of the signal better than the WMA algorithm. However, the signal gets smooth in a very short increments of the parameter change.

            \begin{figure}[ht]
                \centering
                \includegraphics[width=1\textwidth]{images/EMA_ex_2.png} % Replace "path/to/your/image" with the actual path to your image file
                \caption{S\&P 500 signal from 2014 to 2024 overlayed with signal averaged with EMA with $\alpha= 0.06$ on the left side
                and with $\alpha= 0.01$ on the right side.}
                \label{fig:EMA_ex_2}
            \end{figure}

            As can be seen from the figure \ref{fig:EMA_ex_2}, EMA algorithm also tends to lag behind the signal when the
            parameter \begin{math}
                \alpha
                \end{math} is too small.
                \newpage
\subsection{Execution times}
One of the important aspects of the algorithms - execution time savings. 

\begin{table}[ht]
    \centering
    \begin{tabular}{|c|c|c|}
        \hline
        N & WMA & EMA \\
        \hline
        100 &0.0016 &0.0004 \\
        1000 &0.0271 &0.0014 \\
        10000 &0.1709 &0.0052 \\
        50000 &0.6697 &0.0295 \\
        100000 &1.4398 &0.0528 \\
        \hline
    \end{tabular}
    \caption{Execution times for WMA with parameters L = 5 and K = 20 and EMA with parameter $\alpha$ = 0.05 algorithms in seconds}
    \label{tab:execution_times}
\end{table}

As can be seen from the table \ref*{tab:execution_times}, EMA is way faster than
WMA. The execution time of the EMA algorithm is not dependant on the size of the signal, because it only requires the previous value to calculate the new average value. 
The execution time of the WMA algorithm is dependant on the size of the parameter K - size of the window - the bigger the K - more operations to compute in a single iteration. 
This could be deduced from the formula \ref{WMA_equation}.


\begin{table}[ht]
    \centering
\begin{tabular}{|c|c|}
    \hline
    K & Execution time \\
    \hline
    5 & 0.0403 \\
    50 & 0.2524 \\
    100 & 0.4755 \\
    500 & 2.3591 \\
    1000 & 3.3062 \\
    \hline
    
\end{tabular}
\caption{WMA execution time dependance on K parameter. In this example signal with length N = 10000 and parameter L = 5}
\label{tab:WMA_execution_time}
\end{table}
\hspace{1 em} As could be seen in table \ref{tab:WMA_execution_time}, execution time of WMA linearly correlates with 
the size of parameter K.
\begin{figure}[ht]
    \centering
    \includegraphics[width=0.5\textwidth]{images/WMA_exec_time.png} % Replace "path/to/your/image" with the actual path to your image file
    \caption{WMA execution time dependance on K parameter}
    \label{fig:execution_time}
\end{figure}


\newpage
                \subsection{Examples with other signals}

                \hspace{1em} To demonstrate the moving average algorithms versatilty, examples with other
                signals that are based on other domains will be shown. 
                \begin{itemize}
                    \item Signal 1 - ECG signal\cite{ECG} - the signal is just a demonstrattive representation of the ECG signal, i.e. this is not real,but
                    generated data.
                    \begin{figure}[ht]
                        \centering
                        \includegraphics[width=1\textwidth]{images/ECG.png} % Replace "path/to/your/image" with the actual path to your image file
                        \caption{On most left chart the original ECG signal is presented, in the middle chart the ECG signal is averaged with WMA parameters L = 30 K = 150, on the most right, an ECG signal is averaged with EMA $\alpha$ = 0.01}
                        \label{fig:ECG}
                    \end{figure}
                    Due to the length of the signal, only a part of the signal is shown in the figure \ref{fig:ECG}. The ECG signal is a good example of the signal that has a lot of noise, and the moving average algorithms can be used to smooth out the signal.
                   \newpage
                    \item Signal 2 - Personal saving rates in the USA\cite{PSAVERT} - the signal represents the percentage of persons income deduced to savings.
                    \begin{figure}[ht]
                        \centering
                        \includegraphics[width=1\textwidth]{images/PSAVERT.png} % Replace "path/to/your/image" with the actual path to your image file
                        \caption{On most left chart the original signal of personal saving rates in the USA is presented, in 
                        the middle chart the signal is averaged with WMA parameters L = 5 K = 20, on the most right, the signal is averaged with EMA $\alpha$ = 0.05}
                        \label{fig:PSAVERT}
                    \end{figure}

                    As could be seen from the figure \ref{fig:PSAVERT}, the signal when averaged manages retain the
                    trend, however, it cuts short the 2020 peak.
                    \item Signal 3 - weekly petrol prices in Italy\cite{PETROL}
                    \begin{figure}[ht]
                        \centering
                        \includegraphics[width=1\textwidth]{images/FUEL_PRICE_ITALY.png} % Replace "path/to/your/image" with the actual path to your image file
                        \caption{On most left chart the original signal of weekly petrol prices in Italy is presented, in 
                        the middle chart the signal is averaged with WMA parameters L = 24 K = 52, on the most right, the signal is averaged with EMA $\alpha$ = 0.03}
                        \label{fig:PETROL}
                    \end{figure}
                    \newpage
                    \item Singal 4 - random generated signal.                    
                    \begin{figure}[ht]
                        \centering
                        \includegraphics[width=1\textwidth]{images/random_signal.png} % Replace "path/to/your/image" with the actual path to your image file
                        \caption{On most left chart the original random signal is presented, in 
                        the middle chart the signal is averaged with WMA parameters L = 20 K = 25, on the most right, the signal is averaged with EMA $\alpha$ = 0.05}
                        \label{fig:RANDOM}
                        
                    \end{figure}
                    
                \end{itemize}

                \newpage
                \section{Conclusion}

                \hspace{1 em} In this task two algorithms were implemented, WMA and EMA, they were
                analysed, qualitative parameters such as correlation and variation were measured and compared in between. 

                The WMA is more flexible, due to the fact that is has two parameters for optimizations, however,
                some values are lost in analysis, because the averaging is done in range of windows.

                The EMA is more robust, because it does not require all the values in the window to calculate the new average value. 
                The EMA can be used to predict the next upcoming value, if the previous value is known.

                One huge advantage EMA has over WMA - execution time. If the signal is large, the EMA algorithm is way faster than the WMA algorithm.
                However, the EMA is more difficult to fine tune, because it has only one parameter. To increase the
                usability of the EMA, variating parameter $\alpha$ could be used, but this was not in the
                scope of the project.

        \newpage
        \begin{thebibliography}{99}
        \bibitem{SP500} S\&P Dow Jones Indices LLC, S\&P 500 [SP500], retrieved from FRED, Federal Reserve Bank of St. Louis; https://fred.stlouisfed.org/series/SP500, 
        \bibitem{ECG} ECG signal demo; https://www.kaggle.com/datasets/ahmadsaeed1007/ecg-signal?resource=download
        \bibitem{PSAVERT}U.S. Bureau of Economic Analysis, Personal Saving Rate [PSAVERT], retrieved from FRED, Federal Reserve Bank of St. Louis; https://fred.stlouisfed.org/series/PSAVERT
        \bibitem{PETROL}Petrol prices in Italy; https://dgsaie.mise.gov.it/open-data
    \end{thebibliography}
         \end{document}
